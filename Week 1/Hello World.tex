\documentclass[Main.tex]{subfiles}
\begin{document}


\section{Hello World}
    \begin{lstlisting}
    public class Program {
        public static void main (String args[]){
            System.out.println("Hello World!");
        }
    }\end{lstlisting}
    Above is the source file from the previous section. The purpose of this program is to type the
characters Hello World! on the screen.
    A class is a section of a program. (A better definition will come later on in these notes). Small programs often
consist of just one class. When the program is compiled, the compiler will make a file of bytecodes called
Hello.class. Most classes contain many more lines than this one. Everything that makes up a class is placed between the
first brace \{ and its matching last brace  \}.
    \hl{The name for something like a class is called an \textbf{identifier}}.

\end{document}
