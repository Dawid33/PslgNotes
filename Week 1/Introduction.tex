\documentclass[Main.tex]{subfiles}
\begin{document}
    \section{Introduction}

    \begin{lstlisting}
    public class Program {
        public static void main (String args[]){
            System.out.println("Hello World!");
        }
    }\end{lstlisting}

    Here is an example Java program. It is about as small as a Java program can be. When it runs, it writes Hello
World! on the screen. The details will be explained later. This program can be created using a text editor
such as the Notepad editor that comes with Windows. The program will be in a text file on the hard disk, named
Hello.java.

    \hl{A \textbf{source file} is a text file that contains a program (such as above) written in a programming
language.} Since it contains ordinary text (stored as bytes) it can not be directly executed (run) by the computer
system. As a text file, you can print it, display it on the monitor, or alter it with a text editor.

    In order to execute a java program, it must first be compiled. The javac program is a compiler (a translator) that
translates a source file into a file of \textbf{bytecodes}. A Java \textbf{bytecode} is a machine instruction for a
Java processor. Usually, however, people do not have hardware Java processor chips. They have ordinary PCs. A java
processor can be implemented as a program that reads the bytecodes and performs the operations they specify.
This program is called a \textbf{Java Virtual Machine} (JVM).

    If you have already figured out how to create, compile and run Java programs on your computer, you can skip the rest
of this section.
\begin{lstlisting}[language=bash]
    javac Program.java

    java Program\end{lstlisting}
\end{document}
